\section{coulomb2gmt -- pre-released
v1.0-beta5.1}\label{coulomb2gmt-pre-released-v1.0-beta5.1}

\begin{quote}
Bash scripts to plot coulomb results on gmt
\end{quote}

\subsection{Author(s)}\label{authors}

\begin{itemize}

\item
  Demitris G. Anastasiou
\end{itemize}

\begin{center}\rule{0.5\linewidth}{\linethickness}\end{center}

\subsection{Features}\label{features}

\begin{itemize}
\item
  auto-configure map lat-long from input files (.inp)
\item
  Plot Stress changes (Coulomb, Normal, Shear)
\item
  Plot all strain components (E**, Dilatation)
\item
  Plot Fault geometry (Projection, Surface, Depth)
\item
  Plot GPS displacement observed and modeled
\item
  Plot Fault and CMT databases
\item
  Add GMT timestamp logo and custom logo of your organization.
\item
  Adjust paper size to map and convert in different output formats
  (.jpg, .png, .eps, .pdf)
\end{itemize}

\subsection{Requirements}\label{requirements}

\begin{itemize}
\item
  \textbf{GMT}: \href{http://gmt.soest.hawaii.edu/}{The Generic Mappting
  Tools - GMT} version \textgreater{} 5.1.1 . Recommented installation
  from source code.

  \begin{itemize}
  
  \item
    for \emph{Ubuntu/Debian} if you use default package installation you
    have to install also \texttt{libgmt-dev} package
  \end{itemize}
\item
  \textbf{Coulomb 3}:
  \href{https://earthquake.usgs.gov/research/software/coulomb/}{Coulomb
  3, developed by USGS}
\item
  \textbf{python}: required for some calculations included in the main
  script.
\end{itemize}

\subsection{Usage details}\label{usage-details}

The main script is: \texttt{coulomb2gmt.sh}

run:
\texttt{\$\ ./coulomb2gmt.sh\ \textless{}inputfile\textgreater{}\ \textless{}inputdata\textgreater{}\ \textbar{}\ options}

\begin{itemize}
\item
  \texttt{\textless{}inputfile\textgreater{}}: name of input file used
  from Coulomb. Extention \texttt{.inp} not needed. Path to the
  directory of input files configured at \texttt{default-param}.
\item
  \texttt{\textless{}inputdata\textgreater{}}: Name of input files
  include results of coulmb calculations. Input data files are:

  \begin{itemize}
  
  \item
    \texttt{\textless{}inputdata\textgreater{}-gmt\_fault\_surface.dat}:
  \item
    \texttt{\textless{}inputdata\textgreater{}-gmt\_fault\_map\_proj.dat}:
  \item
    \texttt{\textless{}inputdata\textgreater{}-gmt\_fault\_calc\_dep.dat}:
  \item
    \texttt{\textless{}inputdata\textgreater{}-coulomb\_out.dat}:
  \item
    \texttt{\textless{}inputdata\textgreater{}-dcff.cou}:
  \end{itemize}
\end{itemize}

\subsubsection{Default parameters}\label{default-parameters}

Many parameters configured at \texttt{default-param} file. 1. Paths to
general files (DEM, logo, faults) 2. Paths to input file directories
(.inp, .dat, .cou, .disp) 3. ColorMaps Palette, frame variable. 4. Scale
parameters.

\subsubsection{General options}\label{general-options}

\begin{itemize}
\item
  \texttt{-r}: set custom region parameters. \emph{Structure}
  \texttt{-r\ minlon\ maxlon\ minlat\ maxlat\ prjscale}
\item
  \texttt{-topo}: plot topography using dem file
\item
  \texttt{-o\ \textless{}filename\textgreater{}}: set custom name of
  output file. Default is \texttt{\textless{}inputadata\textgreater{}}.
\item
  \texttt{-cmt\ \textless{}path\ to\ file\textgreater{}} : Plot Centroid
  Moment Tensors of earthquakes.
\item
  \texttt{-faults}: Plot custom fault database catalogue.
\item
  \texttt{-mt\ "map\ title"}: Custom map title.
\item
  \texttt{-h}: Help menu
\item
  \texttt{-debug}:Enable Debug option
\item
  \texttt{-logogmt}: Plot GMT logo and time stamp.
\item
  \texttt{-logocus}: Plot custom logo of your organization.
\end{itemize}

\subsubsection{Plot fault parameters}\label{plot-fault-parameters}

\begin{itemize}
\item
  \texttt{-fproj}: Plot source and receiver faults' trace at surface.
\item
  \texttt{-fsurf}: Plot surface of source and receiver faults.
\item
  \texttt{-fdep}: Plot intersection of target depth with fault plane.
\end{itemize}

\subsubsection{Plot stress}\label{plot-stress}

\begin{itemize}
\item
  \texttt{-cstress}: Plot Coulomb Stress change.
\item
  \texttt{-sstress}: Plot Shear Stress change.
\item
  \texttt{-nstress}: Plot Normal Stress change.
\end{itemize}

\subsubsection{Plot Strain components}\label{plot-strain-components}

\begin{itemize}
\item
  \texttt{-stre**}: Where \texttt{**} you can fill all strain components
  \texttt{xx},\texttt{yy},\texttt{zz}, \texttt{yz}, \texttt{xz},
  \texttt{xy}.
\item
  \texttt{-strdil}: Plot dilatation (Exx + Eyy + Ezz )
\end{itemize}

\subsubsection{Plot gps velocities, observed and
modeled}\label{plot-gps-velocities-observed-and-modeled}

\begin{itemize}
\item
  \texttt{-dgpsho}: Observed GPS horizontal displacements.
\item
  \texttt{-dgpshm}: Modeled horizontal displacements on GPS sites (Okada
  1985).
\item
  \texttt{-dgpsvo}: Observed GPS vertical desplacements.
\item
  \texttt{-dgpsvm}: Modeled vertical displacements on GPS sites (Okada
  1985).
\end{itemize}

\begin{quote}
Configure displacement scale in \texttt{default-param} file
\end{quote}

\subsubsection{Output formats}\label{output-formats}

Default format is \texttt{*.ps} file. You can use the options bellow to
convert to other format and adjust paper size to map size.

\begin{itemize}
\item
  \texttt{-outjpg} : Adjust and convert to JPEG.
\item
  \texttt{-outpng} : Adjust and convert to PNG (transparent where
  nothing is plotted)
\item
  \texttt{-outeps} : Adjust and convert to EPS"
\item
  \texttt{-outpdf} : Adjust and convert to PDF
\end{itemize}

\subsection{Contributing}\label{contributing}

\begin{enumerate}
\def\labelenumi{\arabic{enumi}.}

\item
  Create an issue and describe your idea
\item
  \href{https://github.com/demanasta/coulomb2gmt/network\#fork-destination-box}{Fork
  it}
\item
  Create your feature branch (\texttt{git\ checkout\ -b\ my-new-idea})
\item
  Commit your changes
  (\texttt{git\ commit\ -am\ Add\ some\ feature})
\item
  Publish the branch (\texttt{git\ push\ origin\ my-new-idea})
\item
  Create a new Pull Request
\item
  Profit! :white\_check\_mark:
\end{enumerate}

\subsection{ChangeLog}\label{changelog}

The history of releases can be viewed at
\href{docs/ChangeLog.md}{ChangeLog}

\subsection{Acknowlegments}\label{acknowlegments}

\subsection{References}\label{references}

\begin{itemize}
\item
  \href{https://earthquake.usgs.gov/research/software/coulomb/}{Coulomb
  3, developed by USGS}
\item
  Toda, Shinji, Stein, R.S., Sevilgen, Volkan, and Lin, Jian, 2011,
  Coulomb 3.3 Graphic-rich deformation and stress-change software for
  earthquake, tectonic, and volcano research and teaching---user guide:
  U.S. Geological Survey Open-File Report 2011-1060, 63 p., available at
  http://pubs.usgs.gov/of/2011/1060/.
\item
  \href{http://gmt.soest.hawaii.edu/}{The Generic Mappting Tools - GMT}
\end{itemize}
